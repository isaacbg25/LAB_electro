\documentclass[11pt]{article}
\usepackage{graphicx} % Required for inserting images
\usepackage[top=2.5cm, bottom=2.5cm, left=2.5cm, right=2.5cm]{geometry}
\usepackage[T1]{fontenc}
\usepackage{hyperref}
\usepackage[utf8]{inputenc}
\usepackage{multirow}
\usepackage{subcaption}
\usepackage{booktabs}
\usepackage{bookmark}
\usepackage{graphicx}
\usepackage{setspace}
\setlength{\parindent}{0in}
\usepackage{physics}
\usepackage{tikz}
\usepackage{tikz-3dplot}
\usepackage[outline]{contour} % glow around text
\usepackage{xcolor}
\usepackage{float}
\usepackage{makeidx}
\usepackage{fancyhdr}
\usepackage{pgfplots}
\usepackage{amsmath}
\pgfplotsset{compat=1.18}
\usepackage{caption}
\usepackage[english,catalan]{babel}
\setlength{\parskip}{11pt}
\usepackage{xcolor}
\usepackage{listings}
\usepackage{marginnote}
\usepackage{siunitx}


\title{\Huge\bfseries Pràctica 6: \\ Feixos de raigs catòdics \\ [2ex] \Large}

\author{\begin{tabular}{c}
\textbf{GRUP A6} \\
Isaac Baldi García (1667260)\\
Miguel \\
Eira Jacas García (1666616) \\
Victor
\end{tabular}}

\date{Març 2025}

\begin{document}

\maketitle
\begin{center}
    \textbf{Abstract:} 
\end{center}


\newpage

\tableofcontents
\newpage

\section{Introducció Teòrica}
\section{Desviació electroestàtica}\label{sec: desv_electr}
(miguel)
\section{Desviació magnetoestàtica}\label{sec: desv_magn}
En aquest apartat de la pràctica, després de comprovar que les partícules dels ràigs catòdics tenen càrrega negativa com hem vist a la secció \ref{sec: desv_electr} n'estudiem la interecció amb el camp magnètic, substancialment uniforme, generat per unes bobines de Hemholtz. El camp l'hem direccionat perpendicular als raigs catòdics i n'hem controlat la intencitat regulant la intencitat del corrent de les bobines. Al aplicar el camp hem observat com els ràigs es corbaven i ens hem disposat a estudiar les dependències d'aquesta corba i el seu radi amb la intencitat del corrent de les bobines i el potencial dels raigs catòdics per poder determinar més propietats de les partícules d'aquests. 

\begin{equation}
    R=\frac{x^2+y^2}{2y}
    \label{eq: radi}
\end{equation}
----Posar fòrmules camp magnetic i radi------

En primer lloc hem estudiat la dependència del radi de la trajectòria dels ràigs catòdics amb la intencitat del corrent de les bobines agafant $0.1A, 0.2A, 0.3A, 0.4A, 0.5A, 0.6A, 0.7A, 0.8A$ com a intencitats de mostreig. Per això, hem calculat el radi associat a cada intencitat que com veiem a \ref{eq: radi} és el pendent de la regressió linial entre $x^2+y^2$ i $2y$ .

com i per què varia R en funció de I?
(gràfic amb incerteses?)

com i per què varia R en funció de Va?

(!!pensar com evaluar les incerteses)

\section{Desviació electromagnètica}\label{sec: desv_em}

En aquest tercer apartat ens interessem la relació càrrega/massa de les partícules dels raigs catòdics per a comprovar que aquesta coincideix amb la de l'electró. 

Tenint en compte el que ja hem pogut observar en els apartats anteriors: la desviació parabòlica a l'aplicar un camp elèctric E i desviació circular a l'aplicar el camp magnètic B, el que ens interessa en aquest tercer apartat és aplicant els camps de tal manera que de les dues deflexions estiguin al mateix pla però en amb direccions oposades. D'aquesta manera, aconseguim que la trajectòria dels raig catòdics no es vegi desviada, fet que ens permet igualar l forces elèctriques i magnètiques de tal manera quu arribem a la següent equació:

\begin{equation}\label{eq: Fm=Fe}
    qE = qvB
\end{equation}

amb la que es troba que la velocitat vindrà donada pel quocient

\begin{equation}
    v = \frac{E}{B}
\end{equation}

Podem obtenir el radi de la trajectòria circular deguda només a la desviació magnètica com s'explica a la secció \ref{sec: desv_magn}.

De les Eqs \eqref{eq: Fm=Fe}, I MES EQ (LA DEL RADI) es pot deduir l'equació que emprem per a calcular la relació càrrega/massa de la partícula que forma els raigs catòdics a partir de les nostres dades experimentals:

\begin{equation}
    \frac{q}{m}=\frac{E}{RB^2}=\frac{kV_p}{dK^2I^2R}
\end{equation}

Primerament, hem trobat el valor de la diferència de potencial aplicat entre les plaques amb el qual la desviació de la trajectòria rectilina paral·lela a l'eix de les abscisses és mínima\footnote{L'ampliació respecte aquest aspecte es troba en l'annex \ref{sec: traj_no_rect}}. 

En concret hem hagut d'aplicar una diferència de potencial $Vp = (0,85 + )$ kV per compensar un camp magnètic generat per bobines amb intensitat de $I = 100$ mA i un potencial $Va = 3$ kV per a l'accelaració de les partícules dels raigs catòdics.

En segon lloc, suprimint el camp elèctric $\vec{E}$ podem mesurar el radi de la trajectòria que deguda només de la desviació magnètica d'igual manera que en la secció \ref{sec: desv_magn}, el qual ha resultat ser $R = (A + A)$ m.

La constant del condensador $k$, la qual té en compte els efectes de vorada de les plaques, la podem obtenir com hem fet prèviament a la secció \ref{sec: desv_electr}. D'altra banda la constant de les bobines de Hemholtz $K$ ve determinada per la geometria d'aquestes, com s'explica en la secció \ref{sec: desv_magn}.

Per últim, un cop hem trobat la relació càrrega-massa q/m podem comparar-la amb la relació e/m, on e correspon a la càrrega d'un electró i m a la seva massa.

RESULTAT Q/M\footnote{El càlcul de les incerteses associades a aquest resultat es presenten en l'annex \ref{sec: inc_desv_em}}

\section{Conclusions}
les partícules eren indeed electrons!
\appendix{Annex}

\section{askdjfñla}

\section{Càlcul d'incerteses}
\subsection{Desviació electromagnètica}\label{sec: inc_desv_em}

\section{Trajectòria rectilinia amb camp elèctric i camp magnètic aplicat}\label{sec: traj_no_rect}

AFEGIR FOTOS!

Notem que aquesta trajectòria de fet, no és igual de rectílina a la trajectòria que podem observar quan no hi ha aplicat ni camp elèctric ni camp magnètic. Això és degut a la NO UNIFORMITAT DEL CAMP ELÈCTRIC (CONDENSADOR DE PLAQUES PETITES) I A LA NO UNIFORMITAT DEL CAMP MAGNÈTIC (SOLENOIDE NO PERFECTAMENT IDEAL).

\end{document}
