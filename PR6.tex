\documentclass[11pt]{article}
\usepackage{graphicx} % Required for inserting images
\usepackage[top=2.5cm, bottom=2.5cm, left=2.5cm, right=2.5cm]{geometry}
\usepackage[T1]{fontenc}
\usepackage{hyperref}
\usepackage[utf8]{inputenc}
\usepackage{multirow}
\usepackage{subcaption}
\usepackage{booktabs}
\usepackage{bookmark}
\usepackage{graphicx}
\usepackage{setspace}
\setlength{\parindent}{0in}
\usepackage{physics}
\usepackage{tikz}
\usepackage{tikz-3dplot}
\usepackage[outline]{contour} % glow around text
\usepackage{xcolor}
\usepackage{float}
\usepackage{makeidx}
\usepackage{fancyhdr}
\usepackage{pgfplots}
\usepackage{amsmath}
\pgfplotsset{compat=1.18}
\usepackage{caption}
\usepackage[english,catalan]{babel}
\setlength{\parskip}{11pt}
\usepackage{xcolor}
\usepackage{listings}
\usepackage{marginnote}
\usepackage{siunitx}


\title{\Huge\bfseries Pràctica 6: \\ Feixos de raigs catòdics \\ [2ex] \Large}

\author{\begin{tabular}{c}
\textbf{GRUP A6} \\
Isaac Baldi García (1667260)\\
Miguel \\
Eira Jacas García (1666616) \\
Victor
\end{tabular}}

\date{Març 2025}

\begin{document}

\maketitle
\begin{center}
    \textbf{Abstract:} 
\end{center}


\newpage

\tableofcontents
\newpage

\section{Introducció Teòrica}
\section{Desviació electroestàtica}\label{sec: desv_electr}
(miguel)
\section{Desviació magnetoestàtica}
En aquest apartat de la pràctica, després de comprovar que les partícules dels ràigs catòdics tenen càrrega negativa com hem vist a la secció \ref{sec: desv_electr} n'estudiem la interecció amb el camp magnètic, substancialment uniforme, generat per unes bobines de Hemholtz.

com i per què varia R en funció de I?
(gràfic amb incerteses?)

com i per què varia R en funció de Va?

(!!pensar com evaluar les incerteses)
\section{Desviació electromagnètica}
(eira)
\section{Conclusions}
les partícules eren indeed electrons!
\appendix{Annex}

\section{askdjfñla}




\end{document}